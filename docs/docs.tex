%! suppress = TooLargeSection
%! suppress = SentenceEndWithCapital
%! suppress = TooLargeSection
% Preamble
\documentclass[11pt]{PyRollDocs}
\usepackage{textcomp}
\usepackage{csquotes}
\usepackage{wasysym}

\addbibresource{refs.bib}

% Document
\begin{document}

    \title{PyRolL Misaka flow stress Plugin}
    \author{Christoph Renzing}
    \date{\today}

    \maketitle

    This plugin provides the implementation of the constitutive equation from \textcite{Misaka1967} to calculate the flow stress of low alloyed carbon steels.


    \section{Model approach}\label{sec:model-approach}

    The model equation~\ref{eq:misaka-flow-stress} was derived from flow stress measurements.
    Those have been conducted by the author who used hammer drop tests to investigate the flow stress of several low alloyed carbon steels.
    The model equation  takes into account the strain $\epsilon$, strain rate $\dot{\epsilon}$, absolute temperature $T$ as well as the carbon content $C$ of the material.

    \begin{equation}
        k_{\mathrm{f,m}} = \exp{\left( 0.126 - 1.75 C + 0.594 C^2+ \frac{2851 + 2968 C -1120 C^2}{T} \right)} \epsilon^{0.21} \dot{\epsilon}^{0.13}
        \label{eq:misaka-flow-stress}
    \end{equation}

    The equation is valid for strains of up to 0.5, temperatures between \qty{750}{\celsius} and \qty{1200}{\degree\celsius} and strain rates of \qty{30}{\per\second} to \qty{200}{\per\second}.
    The maximum carbon content is 1.2 weight percent.


    \section{Usage instructions}\label{sec:usage-instructions}

    The plugin can be loaded under the name \texttt{pyroll\_misaka\_flow\_stress}.
    The plugin defines the hooks

    \begin{table}[h]
        \centering
        \caption{Hooks specified by this plugin.}
        \label{tab:hookspecs}
        \begin{tabular}{ll}
            \toprule
            Hook name                       & Meaning                                 \\
            \midrule
            \texttt{flow\_stress}           & Flow stress of the material             \\
            \texttt{flow\_stress\_function} & Flow stress as a function of the strain \\
            \bottomrule
        \end{tabular}
    \end{table}

    \printbibliography

\end{document}